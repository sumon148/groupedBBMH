% Options for packages loaded elsewhere
\PassOptionsToPackage{unicode}{hyperref}
\PassOptionsToPackage{hyphens}{url}
%
\documentclass[
]{article}
\usepackage{amsmath,amssymb}
\usepackage{iftex}
\ifPDFTeX
  \usepackage[T1]{fontenc}
  \usepackage[utf8]{inputenc}
  \usepackage{textcomp} % provide euro and other symbols
\else % if luatex or xetex
  \usepackage{unicode-math} % this also loads fontspec
  \defaultfontfeatures{Scale=MatchLowercase}
  \defaultfontfeatures[\rmfamily]{Ligatures=TeX,Scale=1}
\fi
\usepackage{lmodern}
\ifPDFTeX\else
  % xetex/luatex font selection
\fi
% Use upquote if available, for straight quotes in verbatim environments
\IfFileExists{upquote.sty}{\usepackage{upquote}}{}
\IfFileExists{microtype.sty}{% use microtype if available
  \usepackage[]{microtype}
  \UseMicrotypeSet[protrusion]{basicmath} % disable protrusion for tt fonts
}{}
\makeatletter
\@ifundefined{KOMAClassName}{% if non-KOMA class
  \IfFileExists{parskip.sty}{%
    \usepackage{parskip}
  }{% else
    \setlength{\parindent}{0pt}
    \setlength{\parskip}{6pt plus 2pt minus 1pt}}
}{% if KOMA class
  \KOMAoptions{parskip=half}}
\makeatother
\usepackage{xcolor}
\usepackage[margin=1in]{geometry}
\usepackage{graphicx}
\makeatletter
\newsavebox\pandoc@box
\newcommand*\pandocbounded[1]{% scales image to fit in text height/width
  \sbox\pandoc@box{#1}%
  \Gscale@div\@tempa{\textheight}{\dimexpr\ht\pandoc@box+\dp\pandoc@box\relax}%
  \Gscale@div\@tempb{\linewidth}{\wd\pandoc@box}%
  \ifdim\@tempb\p@<\@tempa\p@\let\@tempa\@tempb\fi% select the smaller of both
  \ifdim\@tempa\p@<\p@\scalebox{\@tempa}{\usebox\pandoc@box}%
  \else\usebox{\pandoc@box}%
  \fi%
}
% Set default figure placement to htbp
\def\fps@figure{htbp}
\makeatother
\setlength{\emergencystretch}{3em} % prevent overfull lines
\providecommand{\tightlist}{%
  \setlength{\itemsep}{0pt}\setlength{\parskip}{0pt}}
\setcounter{secnumdepth}{-\maxdimen} % remove section numbering
\usepackage{bookmark}
\IfFileExists{xurl.sty}{\usepackage{xurl}}{} % add URL line breaks if available
\urlstyle{same}
\hypersetup{
  pdftitle={Introduction of groupedBBMH},
  pdfauthor={Sumonkanti Das},
  hidelinks,
  pdfcreator={LaTeX via pandoc}}

\title{Introduction of groupedBBMH}
\author{Sumonkanti Das}
\date{2025-06-12}

\begin{document}
\maketitle

\subsubsection{\texorpdfstring{Overview of the \textbf{groupedBBMH} R
Package}{Overview of the groupedBBMH R Package}}\label{overview-of-the-groupedbbmh-r-package}

The \textbf{groupedBBMH} R package implements a \textbf{nested
beta-binomial hierarchical model} designed for \textbf{group-testing
data in biosecurity surveillance}, where the primary goal is to
efficiently detect rare contamination events. In routine inspections of
imported agricultural consignments, only group-level binary
outcomes---positive or negative---are observed, without information on
the exact number of contaminated items. This package extends traditional
beta-binomial modeling to incorporate \textbf{imperfect test sensitivity
and specificity}, offering both \textbf{exact inference} using a
Metropolis-Hastings (MH) algorithm and \textbf{fast approximate
estimation} via a beta-binomial approximation. It supports estimation of
contamination levels, quantifies the risk of undetected contamination,
and aids in risk-based decision-making through model-based simulation.

\paragraph{Case Study: Frozen Prawn
Importation}\label{case-study-frozen-prawn-importation}

To illustrate the method, consider a biosecurity context involving the
importation of frozen prawns into Australia. Annually, around 800
consignments (batches) are imported, with each batch containing
approximately \(B = 8000\) bags. A simple random sample of \(b = 13\)
bags is tested per batch. From each selected bag, a group of \(m = 5\)
prawns (out of \(M = 40\)) is randomly selected, leading to
\(n = bm = 65\) sampled prawns per batch. Each group of 5 prawns is
tested using a PCR test with sensitivity \(\Delta = 0.70\) and
specificity \(\Lambda = 0.99\). The only observation per batch is the
number of groups testing positive, out of the \(b=13\) tested.

Let \(X_{ij}\) represent the number of contaminated prawns in bag \(j\)
of batch \(i\). Instead of directly observing \(X_{ij}\), we observe
\(Y_{ij}\), where \(Y_{ij} = 1\) if any prawn in the group is
contaminated, and 0 otherwise.

Let define the total contamination as
\(T_{Xi} = \sum_{j=1}^B \sum_{k=1}^M X_{ijk}\), and the sample
contamination as \(t_{xi(m)} = \sum_{j=1}^b \sum_{k=1}^m X_{ijk}\).
Similarly, \(T_{Yi(M)} = \sum_{j=1}^B Y_{ij}\) and
\(t_{yi(m)} = \sum_{j=1}^b Y_{ij}\) are the total and sampled number of
contaminated bags, respectively.

A key quantity of interest is \textbf{leakage}, defined as:

\[
L_i = T_{Xi} \cdot \mathbb{I}(t_{yi(m)} = 0),
\]

representing the number of contaminated prawns entering undetected when
no sampled group tests positive. We focus on:

\begin{itemize}
\tightlist
\item
  the \textbf{expected leakage} \(\mathbb{E}(L_i)\),
\item
  the \textbf{probability of leakage} \(\Pr(L_i > 0)\),
\item
  and posterior inference on \(T_{Xi}\) when \(t_{yi(m)} = 0\).
\end{itemize}

Now let \(p_i\) denote the true prevalence of contamination in batch
\(i\), and define \(\phi_{i(m)} = 1 - (1 - p_i)^m\) as the probability
that a pool of \(m\) (\(\ 1 \le m \le M\)) prawns is contaminated.
Assuming prawns are randomly distributed among bags, we have:

\[
X_{ij} \mid p_i \overset{\text{i.i.d.}}{\sim} \text{Bin}(M, p_i), \quad p_i \overset{\text{i.i.d.}}{\sim} \text{Beta}(\alpha, \beta).
\]

Thus, the group-level test outcome \(Y_{ij}\) follows:

\[
Y_{ij} \mid p_i \sim \text{Bernoulli}(\phi_i), \quad \phi_i = 1 - (1 - p_i)^m,
\]

and the observed number of positive groups in the sample is:

\[
t_{yi} \mid p_i \sim \text{Binomial}(b, \phi_i).
\]

To incorporate \textbf{imperfect testing}, we define the effective
probability of a positive test as:

\[
\tilde{\phi}_{\Delta\Lambda}(p_i) = \Delta \phi(p_i) + (1 - \Lambda)(1 - \phi(p_i)).
\]

Under perfect specificity (\(\Lambda = 1\)), this simplifies to:

\[
\tilde{\phi}_\Delta(p_i) = \Delta \phi(p_i).
\]

When \(\beta \gg \alpha\), the contamination prevalence \(p_i\) is
approximately Gamma distributed, leading to:

\[
\tilde{\phi}_{\Delta\Lambda} \sim (1 - \Lambda) + \text{Beta} \left( \alpha, \frac{\beta}{m(\Delta + \Lambda - 1)} \right).
\]

Two important special cases:

\begin{itemize}
\tightlist
\item
  \textbf{Perfect testing} (\(\Delta = 1\), \(\Lambda = 1\)):
  \(\tilde{\phi} \sim \text{Beta}(\alpha, \beta / m)\),
\item
  \textbf{Perfect specificity} (\(\Lambda = 1\)):
  \(\tilde{\phi} \sim \text{Beta}(\alpha, \beta / (m \Delta))\).
\end{itemize}

In either case, \(t_{yi}\) approximately follows a Beta-Binomial
distribution:

\[
t_{yi} \sim \text{Beta-Binomial}(b, \alpha, \beta / (m\Delta)).
\]

\paragraph{Threshold-Based Risk
Estimation}\label{threshold-based-risk-estimation}

In the context of the prawn biosecurity study, suppose a regulatory
threshold is set such that contamination levels below a certain
prevalence cut-off are considered acceptable for import. Suppose a
regulatory cut-off \(k\) is introduced, such that contamination
prevalence below \(k\) is considered acceptable. The effective
prevalence becomes:

\[
p_i = p_i \cdot \mathbb{I}(p_i > k), \quad p_i \sim \text{Beta}(\alpha, \beta).
\]

Using this truncated model, the \textbf{probability of leakage} becomes:

\[
\Pr[L_i > 0] = \frac{B_{(k_1,1)}(\alpha, \frac{\beta}{m\Delta} + b)}{B(\alpha, \frac{\beta}{m\Delta})} - \frac{B_{(k,1)}(\alpha, \beta + MB)}{B(\alpha, \beta)},
\]

with \(k_1 = \Delta(1 - (1 - k)^m)\). Under perfect testing:

\[
\Pr[L_i > 0] = \frac{B_{(k,1)}(\alpha, \beta + mb)}{B(\alpha, \beta)} - \frac{B_{(k,1)}(\alpha, \beta + MB)}{B(\alpha, \beta)}.
\]

When \(k = 0\), this reduces to the standard beta-binomial case.

The \textbf{expected leakage} under threshold \(k\) is:

\[
\mathbb{E}(L_i) = (B - b) M \cdot \mathbb{E}\left[ (1 - p_i)^{bm\Delta} \cdot p_i \cdot \mathbb{I}(p_i > k) \right],
\]

which simplifies to:

\[
\mathbb{E}(L_i) = (B - b) M \cdot \frac{B_{(k,1)}(\alpha + 1, \beta + bm\Delta)}{B(\alpha, \beta)}.
\]

\end{document}
